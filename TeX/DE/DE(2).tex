\documentclass{article}	
\usepackage[margin=1in]{geometry}
\usepackage{fancyhdr}
\usepackage{duckuments}
\usepackage{enumitem}
\usepackage{graphicx}
\graphicspath{C:\Users\panda\.vscode\tex\MathfuerEcoII\MSE}
\usepackage{hyperref}
\pagestyle{fancy}
\fancyhead{} % clear all header fields
\fancyhead[LO]{\small{Math für Natur-und Wirtschaftswissenschaften}}
\fancyhead[RO]{\small{Eigen vectors}}
\fancyfoot{} % clear all footer fields
\usepackage{amsmath}
\begin{document}
\section*{Basics of Complex Numbers}
A complex number is represented as $Z = a + i b$ where a and b are real numbers and \begin{equation*} Re(Z)=a \end{equation*} \begin{equation*} Im(Z)=b \end{equation*}
\begin{equation*} i = \sqrt{-1} \end{equation*}
\begin{equation*} i^{2} = -1 \end{equation*}
A complex number can be represented on an argand plane with the X-axis as the real part of the complex number and the Y-part as the imaginary part of the complex number\\[2pt]
\begin{figure}[h]
    \centering
    \includegraphics[scale=0.5]{argandplane.png}
    \end{figure}
\subsection*{Algebra of Complex numbers}
Two complex numbers $Z_1 = a_1 + i b_1$ and $Z_2 = a_2 + i b_2$ can be added and subtracted by seperately adding/subtracting their real and imaginary parts. \\
Multiplication of two complex numbers,
\begin{equation*} Z_1 \cdot Z_2 = a_1 a_2 + i a_1 b_2 + i b_1 a_2 + i^{2} b_1 b_2 = a_1 a_2 + i (a_1 b_2 + b_1 a_2) - b_1 b_2  \end{equation*}
Division of two complex numbers, \\
Mulitiplying both numerator and denominator by the \textbf{conjugate} of the denominator inorder to obtain a real denominator and a complex numerator.
For example,
\begin{equation*} \frac{4+3i}{1+2i} = \frac{(4+3i)(1-2i)}{(1+2i)(1-2i)} = \frac{4-8i+3i-6i^2}{1-4i^2} = \frac{12+5i}{5} = \frac{12}{5} + i \end{equation*}
\subsection*{Complex Conjugate and absolute value}
For a complex number $Z = a + ib$, the complex conjugate is defined as $\overline{Z} = a - ib$ 
\begin{equation} Z\overline{Z} = (a+ib)(a-ib) = a^{2} - i ab + i ab + b^{2} = a^{2} + b^{2}\end{equation}
As per the definition of the absolute value of a complex number
\begin{equation} |Z| = \sqrt{a^2 + b^2} \end{equation} 
By (1) and (2), we have
\begin{equation*} Z\overline{Z} = |Z|^{2} \end{equation*}
\subsection*{Polar form and Euler's form}
Representing the complex number in the form of a $|Z|$ and argument $\phi$ 
\begin{equation*} Z = |Z| (\cos\phi+i \sin\phi)\end{equation*} is the Polar representation of a complex number
\begin{equation*} e^{i \phi} = \cos\phi + i \sin \phi \end{equation*}
Therefore, 
\begin{equation*} Z = |Z| e^{i \phi} \end{equation*} is the Euler form of the complex number 
A lot of cool stuff related to \href{https://www.youtube.com/playlist?list=PLJbzH0qGCsyrO7lUmdEQAzJInNlwwUgp0}{complex number} on this playlist. Not required for economic aspects though.
\section*{Eigen Vectors and Values}
$S \in R^{n\times n}$ \\
$A \in R^{n \times n}$ \\
$S^{-1}AS = D$ diagonal matrix, tranformation matrix \\
Is there a basis transformation? \\ 
$\lambda \in R$ is called an eigenvalue of A if there exists a vector $v \neq \phi$ such that \begin{equation*} A \cdot v = \lambda \cdot v \end{equation*}
Any such vector v in this case would be called an eigen vector of A for the eigen value $\lambda$
\textbf{Gaussian Reformulation:} Determination of the eigenvalues and eigenvectors. \\
$\lambda$ is an eigenvalue of A if there $v\neq\phi$ with $A\cdot v = \lambda \cdot v$
\begin{equation*}
    \Leftrightarrow A\cdot v - \lambda \cdot v = 0 \end{equation*}
\begin{equation*}\Leftrightarrow (A-\lambda E_n) \cdot v = 0 \end{equation*}
\begin{equation*} \Leftrightarrow v \in \ker(A\cdot\lambda E_n) \end{equation*}
\begin{equation*}\Leftrightarrow (A-\lambda E_n) \text{  is not invertible} \end{equation*}
\begin{equation*} \Leftrightarrow \det(A-\lambda E_n) = 0 
\end{equation*} \\[2pt]
The \textbf{characteristic polynomial} $p_A(\lambda)$ is defined as the determinant $\det(A-\lambda E_n)$ \begin{equation*} p_A(\lambda) = \det(A-\lambda E_n) \end{equation*} \\ and the kernel $\ker(A-\lambda E_n)$ is called the \textbf{eigenspace of A for the eigenvalue $\lambda$} and written as \begin{equation*} eig_A(\lambda) = \ker(A-\lambda E_n) \end{equation*}
The idea to solve each and every problem here is quite simple,
\begin{enumerate}
    \item First of all, determine the \textbf{characteristic polynomial} [Diagonalized matrix S is the matrix with the eigenspaces of the respective eigenvalues]
    \item Find the roots of the characteristic polynomial, these are the eigenvalues of A
    \item For each eigenvalue of A, determine the eigenspace by solving the linear equation system \\ $(A-\lambda E_n) \cdot v = 0$ 
\end{enumerate}
\subsection*{Diagonalization using complex numbers} 
characterizing which matrices can be transformed into diagonal form, ie possess an eigenvector basis. \\
A matrix $A\in R^{n\times n}$ is called \textbf{diagonalizable} if there is a basis of $R^{n}$ that consists of eigenvectors of A\\[2pt]
\underline{NOTE:} For each eigenvalue $\rightarrow$ there is at least one eigenvector ie. $\dim(eig_A(\lambda)) \geq 1$ for each eigenvalue $\lambda$ 
$\Rightarrow$ if A possesses n distinct eigenvalue, then there are at least n eigenvectors \\
\begin{equation*} p_A(\lambda) = {(\lambda - \lambda*)}^k \cdot g(\lambda) \text{   and  } g(\lambda*) \neq 0 \end{equation*}
Then k is called the \textbf{algebraic multiplicity} of $\lambda*$ and $\dim(eig_A(\lambda*))$ is called the geometric mutliplicity of $\lambda*$
\begin{enumerate}
    \item Algebraic mulitplicity $\geq$ Geometric mulitplicity
    \item For pairwise distinct eigenvalues of A with corresponding eigenvectors- the set $\{v_1, \dots, v_r\}$ is linearly independent
    \item For a matrix to be diagonalizable, the algebraic and the geometric mulitplicity have to be equal for every eigenvalue of A 
\end{enumerate}
Proving (2) is pretty simple as you can simply use the principle of induction after solving it for 2 eigenvalues and their corresponding eigenvectors
\\ If roots are complex, we can use the Euler's formula \begin{equation*} e^{i \phi} = \cos\phi + i \sin \phi \end{equation*}
what if the matrix cannot be diagonalized? \textbf{JORDAN NORMAL FORM} as \href{https://en.wikipedia.org/wiki/Jordan_normal_form}{generalizations} and refer \href{https://shorturl.at/9Vxni}{for} Inhomogeneous Differential Equation system 
\\\url{https://people.math.harvard.edu/~knill/teaching/math19b_2011/handouts/lecture29.pdf} for a recap
\subsection*{Predator-Prey Model}
squirrels-hawks $\rightarrow$ The growth of hawk population depends on availability of squirrels and vice-versa. [Fewer hawks $\rightarrow$ higher the growth rate of squirrels]
\\ Model this through DE system:
\begin{equation*} h'(t) = s(t) - 12 \text{ and } h(0) = 6\end{equation*}
\begin{equation*} s'(t) = -h(t) \text{  and  } s(0) = 20 \end{equation*}
\begin{enumerate} 
    \item \underline{get rid of constants:} define a new system \\
    \begin{equation*} y_1(t) = h(t) - 10 = -s'(t) \text{    and    } y_1(0) = h(0) - 10 = -4\end{equation*}
    \begin{equation*} y_2(t) = s(t) - 12 = h'(t) \text{   and   }  y_2(0) = s(0) - 12 = 8 \end{equation*}
    \begin{equation*} \Rightarrow y_1'(t) = h'(t) = s(t) - 12 = y_2(t)  \end{equation*}
    \begin{equation*} \Rightarrow y_2'(t) = -h(t) + 10 \end{equation*} 
    \text{new system without constants:} 
    \begin{equation*} \begin{pmatrix} y_1'(t) \\ y_2'(t) \end{pmatrix} = \begin{pmatrix} 0 & 1 \\ -1 & 0 \end{pmatrix} \begin{pmatrix} y_1(t) \\ y_2(t) \end{pmatrix} \mid \begin{pmatrix} y_1(0) \\ y_2(0) \end{pmatrix} = \begin{pmatrix} -4 \\ 8 \end{pmatrix} \end{equation*} 
    \item \underline{Diagonalize the matrix A} [find eigenvalues and the corresponding eigenvectors then form the S matrix]\begin{equation*} A = \begin{pmatrix} 0 & 1 \\ -1 & 0 \end{pmatrix} \end{equation*}
    not showing the calculations but the eigenvectors are $+i$, $-i$ and eigenvectors are $\begin{pmatrix} 1 \\ i \end{pmatrix}$ and $\begin{pmatrix} 1 \\ -i \end{pmatrix}$ so the required S matrix is $\begin{pmatrix} 1 & 1 \\ i & -i \end{pmatrix}$ \\
    \begin{equation*} S^{-1}AS = \begin{pmatrix} i & 0 \\ 0 & -i \end{pmatrix} \end{equation*}
    \item \underline{consider the decoupled system after substitution} $y=SZ$ \\
    \begin{equation*} Z'(t) = \begin{pmatrix} i & 0 \\ 0 & i \end{pmatrix} Z(t)\end{equation*}
    $\Rightarrow$
    \begin{equation*} Z_1'(t) = i\cdot Z_1(t) \Rightarrow Z_1(t) = \gamma_1 \cdot e^{i t} \end{equation*}
    \begin{equation*} Z_2'(t) = -i \cdot Z_2(t) \Rightarrow Z_2(t) = \gamma_2 \cdot e^{-i t} \end{equation*}
    \item \underline{determine y = SZ:} \begin{equation*} y(t) = \begin{pmatrix} 1 & 1 \\ i & -i \end{pmatrix} \begin{pmatrix} \gamma_1 e^{i t} \\ \gamma_2 e^{-i t} \end{pmatrix}\end{equation*}  
    $\Rightarrow$
    \begin{equation*} y_1(t) = \gamma_1 e^{i t} + \gamma_2 e^{-i t} \end{equation*}
    \begin{equation*} y_2(t) = \gamma_2 i e^{i t} - \gamma_2 i e^{-i t} \end{equation*} 
    \item \underline{Determine $\gamma_1$, $\gamma_2$ through initial values:}
    \begin{equation*} y_1(0) = -4 = \gamma_1 \cdot 1 + \gamma_2 \cdot 1 \end{equation*}
    \begin{equation*} y_2(0) = 8 = \gamma_1 \cdot i \cdot 1 - \gamma_2 \cdot i \cdot 1 \end{equation*}
    \begin{equation*} \rightarrow \begin{pmatrix} 1 & 1 &\bigm| & -4 \\  i & -i &\bigm| & 8  \end{pmatrix} \rightarrow \begin{pmatrix} 1 & 1 &\bigm| & -4 \\ 0 & -2i &\bigm| & 8+4i \end{pmatrix} \end{equation*}
    \begin{equation*} (-2i) \gamma_2  = 8 + 4 i \end{equation*}
    \begin{equation*} \Leftrightarrow \gamma_2 = 4i - 2 = -2 + 4i \end{equation*} 
    \begin{equation*} \Rightarrow \gamma_1 = -4 - \gamma_2 = -2 - 4i \end{equation*}
    \textbf{Solutions:}
    \begin{equation*} y_1(t) = (-2 - 4i) e^{i t} + (-2 + 4i) e^{-i t} \end{equation*} 
    \begin{equation*} y_2(t) = (4 - 2i) e^{i t} + (4 + 2i) e^{-i t} \end{equation*}
    \item \underline{Use Euler's Formula:} replace $e^{i t}$ and $e^{-i t}$ with $cis\theta$ terms
    \begin{equation*} y_1(t) = -4\cos t + 8\sin t \mid h(t) = -4\cos t + 8\sin t + 10\end{equation*}
    \begin{equation*} y_2(t) = 8\cos t + 4\sin t \mid s(t) = 8\cos t + 4\sin t + 12 \end{equation*}
\end{enumerate} 
\end{document}
